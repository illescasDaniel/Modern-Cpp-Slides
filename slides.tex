\documentclass{beamer}

\usetheme{metropolis}
\usepackage{listings} % Código
%\usepackage[utf8]{inputenc}
\usepackage[normalem]{ulem} % Tachar texto (con \sout)
\usepackage[english]{babel}
\usepackage{hyperref}
\usepackage{color}

\definecolor{dkgreen}{rgb}{0,0.52,0}
\definecolor{gray}{rgb}{0.5,0.5,0.5}
\definecolor{mauve}{rgb}{0.4,0,0.7}
\definecolor{customBlue}{rgb}{0.1,0.1,0.65}
\definecolor{customOrange}{rgb}{0.88,0.53,0.13}

% Cambiar tipo letra
%\setsansfont{Arial}

\usepackage{fontspec}
\setmonofont[
  Contextuals={Alternate}
]{Fira Code}

\lstset{frame=tb,
  language=c++,
  aboveskip=5mm,
  belowskip=5mm,
  showstringspaces=false,
  columns=flexible,
  basicstyle={\scriptsize\ttfamily},
  numbers=left, 
  numberstyle=\tiny, 
  stepnumber=1, 
  numbersep=8pt,
  numberstyle=\tiny\color{gray},
  keywordstyle=\color{customBlue},
  commentstyle=\color{dkgreen},
  stringstyle=\color{mauve},
  breaklines=true,
  breakatwhitespace=true,
  tabsize=4,
  gobble=16
}

\lstset{literate=
    {->}{{{\texttt{-> }}}}1
    {<-}{{{\texttt{<- }}}}1    
    {<->}{{{\texttt{<-> }}}}1
}

%\lstset{literate=%
%    *{0}{{{\color{customOrange}0}}}1
%    {1}{{{\color{customOrange}1}}}1
%    {2}{{{\color{customOrange}2}}}1
%    {3}{{{\color{customOrange}3}}}1
%    {4}{{{\color{customOrange}4}}}1
%    {5}{{{\color{customOrange}5}}}1
%    {6}{{{\color{customOrange}6}}}1
%    {7}{{{\color{customOrange}7}}}1
%    {8}{{{\color{customOrange}8}}}1
%    {9}{{{\color{customOrange}9}}}1
%}

\newcommand{\normalSizeItem}[1] {
  \normalsize{\item #1}
}

\newcommand{\newFrameWithoutIndex}[1]{
	\begin{frame}
		#1
		\thispagestyle{empty}
	\end{frame}
}

\newcommand{\newSectionWithoutIndex}[1]{
	\newFrameWithoutIndex{\section{#1}}
}

\newcommand{\urlItem}[1]{
	\footnotesize{\item {\url {#1}}}
}

\newcommand{\tinyUrlItem}[1]{
	\tiny{\item {\url {#1}}}
}
  
\newcommand{\smallCite}[1]{
	\begin{small}
		\cite{#1}	
	\end{small}
}

\title{Modern C++}
\author{Daniel Illescas Romero}
\date{\today}
\institute{University Of Granada [UGR]}

\setcounter{tocdepth}{1} % 2 para mostrar subsecciones
\setcounter{secnumdepth}{2}
\setbeamerfont{subsection in toc}{size=\tiny}

\begin{document}

	\newFrameWithoutIndex{\maketitle}
	
	\begin{frame}{Index}		
		\setbeamertemplate{section in toc}[sections numbered]
		\tableofcontents
	\end{frame}

	\newSectionWithoutIndex{Style \& Good practices}
  
		\begin{frame}[fragile]{Naming convention}	
			\subsection{Naming convention}
			\begin{itemize}
			
				\normalSizeItem {Bad idea}
				\begin{lstlisting}
				int main() {
					int a, b;
					double c;
				}
				\end{lstlisting}
				
				\normalSizeItem {Good idea}
				\begin{lstlisting}
				int main() {
					unsigned int age = 20;
					int points = 0;
					float height = 1.7;
				}
				\end{lstlisting}
				
			\end{itemize}
		\end{frame}
			 
		\begin{frame}[fragile]{Naming convention}	
			\begin{itemize}
			
				\normalSizeItem{Best idea \textbf{[C++11]}}
				\begin{lstlisting}
				int main() {
					uint8_t age{10};
					int64_t points{}; 
					float height{};
				}
				\end{lstlisting}
				
				\normalSizeItem{Freak! \textbf{[C++11]}}
				\begin{lstlisting}
				int main() {
					auto age = uint8_t{20};
					auto points = int64_t{0};
					auto height = float{5.67};
				}
				\end{lstlisting}
				
			\end{itemize}
		\end{frame}
		
		\begin{frame}[fragile]{Programming styles}	
			\subsection{Programming styles}
			\begin{itemize}
			
				\normalSizeItem{Variable names \smallCite{namingConvention}}
				\begin{lstlisting}
				// Delimiter separated words
				float daniel_height = 1.7;
				// CamelCase (letter-case separated words)
				float johnHeight = 1.7;
				\end{lstlisting}
				
				\normalSizeItem{Braces (use it always!) \smallCite{indentationStyle}}
				\begin{lstlisting}
				// K&R (Kernighan and Ritchie)
				if () {
				
				}
				// Allman
				if ()
				{
				
				}
				\end{lstlisting}
				
			\end{itemize}
		\end{frame}
		
		\begin{frame}[fragile]{Macros \& global variables}	
			\subsection{Macros \& global variables}
			\begin{itemize}
			
				\normalSizeItem{Avoid the use of macros}
				\begin{lstlisting}
				// Type?, Scope?
				// Is not constant and can be undefined
				#define PI 3.14159
				#define SIZE 10
				
				int main() {
					
					// float PI = 10; -> Would be: float 3.14159 = 10 ¿?
					
					int numbers[SIZE];
					
					#undef SIZE
					
					// Error: "SIZE" is not declared...
					cout << SIZE << endl;
				}
				\end{lstlisting}
				
			\end{itemize}
		\end{frame}
		
		\begin{frame}[fragile]{Macros \& global variables}	
			\begin{itemize}
			
				\normalSizeItem{Try NOT to use global variables}
				\begin{lstlisting}[basicstyle={\tiny\ttfamily}]
				#include <vector>
				#include <array>
				
				// It leads to many problems... BAD IDEA
				// Ambiguity ERROR with: std::__1::size
				int size = 20; // Can a size be negative?
				
				namespace utils { // This would be a safe alternative
					constexpr size_t arraySize = 20;
				}
				
				// How do I pass an array? You don't know the size here
				void doSomething(int arr[]) {}
				
				void doSomething(int arr[size]) {} // NO...
				
				// Valid option
				void doSomething(int* arr, const size_t arrSize) {}
				
				// If you know it's size
				void doSomething(std::array<int,20> arr) {}
				
				// If you DON'T know it's size
				void doSomething(std::vector<int> arr) {}
				\end{lstlisting}
				
			\end{itemize}
		\end{frame}
		
		\begin{frame}[fragile]{Platform dependency}	
			\subsection{Platform dependency}		
			\begin{itemize}

				\normalSizeItem { The use of \texttt{system} can lead to many errors}
				\begin{lstlisting}
				// It doesn't work in the Windows command line (CMD)
				std::system("ls -l > test.txt");
				
				// Only works in Windows :/
				std::system("pause"); 
				\end{lstlisting}
				
				\item { 
					
					Some solutions: 
								
					\begin{itemize}
						\item {
							We can use this class to list all files in a folder:
							 \begin{verbatim}
							 	std::filesystem::recursive_directory_iterator
							 \end{verbatim}
						}
						\item {
							To pause the program execution: \texttt{cin.ignore();}
						}
					\end{itemize}
				}
			\end{itemize}
		\end{frame}
		
		\begin{frame}[fragile]{Platform dependency}	
			\subsection{Platform dependency}		
			\begin{itemize}
				\normalSizeItem { If you use old and dependent libraries you can't get \textbf{portable} code \smallCite{posixThreads}}
						\begin{lstlisting}[basicstyle={\tiny\ttfamily}]
						#include <pthread.h>
		
						void* print_message_function(void* ptr) {
							char *message;
							message = (char *) ptr;
							printf("%s \n", message);
						}
						
						int main() {
							int iret1 = pthread_create(&thread1, NULL, print_message_function, (void*) message1);
						}
					\end{lstlisting}
			\end{itemize}
		\end{frame}
		
		\begin{frame}[fragile]{Platform dependency}	
			\begin{itemize}

				\normalSizeItem { Try to use the own language libraries, or at least updated ones }
				\begin{lstlisting}[basicstyle={\tiny\ttfamily}]
				#include <thread>
				
				void doSomething(int number) { ... }
				void doSomething2(int& number) { ... }
				
				int main() {
				
					int number = 0;
					
					// Pass by value
					std::thread thread1(doSomething, number + 1); 
					
					// Pass by reference
					std::thread thread2(doSomething2, std::ref(number)); 
					
					// thread3 is running "doSomething2()", thread2 is no longer a thread
					std::thread thread3(std::move(thread2)); 
					
					thread1.join();
					thread3.join();					
				}
				
				\end{lstlisting}
				
			\end{itemize}
		\end{frame}
		
		\begin{frame}[fragile]{Semantic Versioning \smallCite{semanticVersioning}}	
			
			\textbf{Naming:} MAJOR.MINOR.PATCH
			\newline\newline
			Example: \textbf{1.3.4}
		
			Increment the version when you:
			\begin{itemize}
				\normalSizeItem { \textbf{MAJOR}: make incompatible API changes}
				\normalSizeItem { \textbf{MINOR}: add functionality in a backwards-compatible manner}
				\normalSizeItem { \textbf{PATH}: make backwards-compatible bug fixes}
			\end{itemize}
		\end{frame}
		
		\begin{frame}[fragile]{Semantic Versioning}	
			
			How you should do it:
			\begin{itemize}
				\item Start with the \textbf{0.1} version
				\item While the SW is in development increase the \textbf{0.x.y }version
				\item Use \textbf{v1.0.0} for production
				\item Try not to use really big numbers
			\end{itemize}
		\end{frame}
		
		\newSectionWithoutIndex{New in C++}	
		
		\begin{frame}[fragile]{\texttt{auto}, type inference -\smallCite{cppReference}}\subsection{\texttt{auto}, type inference}	
			\begin{itemize}
			
				\normalSizeItem{Basic types}
				\begin{lstlisting}
				auto number = 2; // int
				auto width = 0.5; // double
				auto name = "daniel"; // const char*
				auto name = "daniel"s; // string
				\end{lstlisting}
				
				\normalSizeItem{More complex ones}
				\begin{lstlisting}[basicstyle={\tiny\ttfamily}]
				vector<int> points{1,2,3};
				
				for (vector<int>::iterator it = points.begin(); it != points.end(); ++it) {
					cout << *it << endl;
				}
				for (auto it = points.begin(); it != points.end(); ++it) {
					cout << *it << endl;
				}
				\end{lstlisting}
				
			\end{itemize}
		\end{frame}
		
		\begin{frame}[fragile]{\texttt{auto}, type inference}	
			\begin{itemize}
			
				\normalSizeItem{Can be used also in functions with large type names}
				\begin{lstlisting}[basicstyle={\tiny\ttfamily}]
				std::chrono::time_point<std::chrono::high_resolution_clock> now() {
					return std::chrono::high_resolution_clock::now();
				}
				
				auto now() {
					return std::chrono::high_resolution_clock::now();
				}
				\end{lstlisting}
								
				\normalSizeItem { Other ways }
				\begin{lstlisting}[basicstyle={\tiny\ttfamily}]
				auto add(int value, double value2) -> decltype(value + value2) {
					return value + value2;
				}
				auto calculateWeight(...) -> double { ... }
				\end{lstlisting}
				
			\end{itemize}
		\end{frame}
		
		\begin{frame}[fragile]{Fixed width integers types [C++11] \& \textit{brace initializer}}	
			\subsection{Fixed width integers types [C++11] \& ``brace initializer''}
			\begin{itemize}
			
				\normalSizeItem{Main types}
				\begin{lstlisting}[basicstyle={\tiny\ttfamily}]
				#include <cstdint>

				int8_t, uint8_t  // Signed/ Unsigned type with 8 bits
				int16_t, uint16_t // 16 bits
				int32_t, uint32_t  // 32 bits
				int64_t, uint64_t // 64 bits
				intmax_t, uintmax_t // Largest capacity (usually 64 bits)
				size_t // To represent array capacities [Unsigned 64 bits integer]
				\end{lstlisting}
								
				\normalSizeItem { Correctly initialising types }
				\begin{lstlisting}
				uint64_t bigStuff {643456787654};
				int32_t things = {3500000000};
				// uint8_t age {-20}; // Compilation error
				things += {2000000000};
				// things = 1205032704 :(
				// you can't avoid overflow...
				\end{lstlisting}
				
			\end{itemize}
		\end{frame}
		
		\begin{frame}[fragile]{\sout{NULL}, \ttfamily{nullptr}}	
			\subsection{\sout{NULL}, \ttfamily{nullptr}}
			\begin{itemize}
			
				\normalSizeItem { NULL can lead to errors! }
				\begin{lstlisting}[basicstyle={\tiny\ttfamily}]
				void doSomething(int* ptrI) { ... }
				void doSomething(double* ptrD) { ... }
				void doSomething(std::nullptr_t nullPointer) { ... } 
				
				int main() {
				
					int* ptrI;
					double* ptrD;
				 
					doSomething(ptrI);
					doSomething(ptrD);
					doSomething(nullptr);  // ambiguous without void f(nullptr_t)
					//doSomething(NULL);  // ambiguous: all functions are candidates
				}
				\end{lstlisting}
				
				\normalSizeItem { Assign null}
				\begin{lstlisting}
				int* something = NULL; // before C++11: NULL = 0
				double* foo = nullptr; // C++11: std::nullptr_t
				\end{lstlisting}
				
			\end{itemize}
		\end{frame}
		
		\begin{frame}[fragile]{Alternative to \textit{raw pointers}. \ttfamily{unique\_ptr<int>}}	
			\subsection{Alternative to raw pointers. \ttfamily{unique\_ptr<int>}}
			\begin{itemize}
			
				\normalSizeItem { Automatic Memory Management }
				\begin{lstlisting}[basicstyle={\tiny\ttfamily}]
				#include <memory>
				
				using namespace std;
				
				int* giveMeAnInt() { 
					return new int{100};
				}
				
				unique_ptr<int> giveMeACoolInt() {
					return unique_ptr<int>(new int{100});
				}
				
				auto giveMeTheBestInt() {
					return make_unique<int>(100);
				}
				
				int main() {	
					// unique_ptr automatically releases the memory, int* don't
					auto number = giveMeTheBestInt();
					cout << *number << endl;
					
					int* otherNumber = giveMeAnInt();
					delete otherNumber;
				}
				\end{lstlisting}				
			\end{itemize}
		\end{frame}
		
		\begin{frame}[fragile]{Alternative to \textit{raw pointers}. \ttfamily{unique\_ptr<int>}}	
			\begin{itemize}
			
				\normalSizeItem { Automatic Memory Management }
				\begin{lstlisting}
				#include <iostream>
				#include <memory>
				
				using namespace std;
				
				int main() {	
					
					constexpr size_t arrSize { 1000 };
					
					unique_ptr<int[]> numbers(new int[arrSize]);
					//[C++14] auto numbers = make_unique<int[]>(arrSize);
					
					fill(&numbers[0], &numbers[arrSize], 10);
					
					for (size_t i = 0; i < arrSize; i++) {
						cout << numbers[i] << endl;
					}
				}
				\end{lstlisting}				
			\end{itemize}
		\end{frame}
		
		\begin{frame}[fragile]{C++ Castings}	
			\subsection{C++ Castings}		
			\begin{itemize}
			
				\normalSizeItem { Normal Castings }
				\begin{lstlisting}
				int number = 100;
				float height = (int)number; // C style
				height = int(number); // C++ style
				number = static_cast<int>(3.14);
				// const_cast, dynamic_cast, reinterpret_cast
				\end{lstlisting}
				
				\normalSizeItem { Implicit and explicit conversions }
				\begin{lstlisting}[basicstyle={\tiny\ttfamily}]
				struct Foo {
					// implicit conversion
					operator int() const { return 7; } 
					// explicit conversion
					explicit operator int*() const { return nullptr; }   
					explicit Foo(size_t elementsCount) { ... }
				...
				Foo x;
				int* q = x; // Error
				\end{lstlisting}
				
			\end{itemize}
		\end{frame}
		
		\begin{frame}[fragile]{Range-based for loop. I/O manipulators}	
			\subsection{Range-based for loop. I/O manipulators}		
			\begin{itemize}
			
				\normalSizeItem { Modern for-loop for collections [C++11] }
				\begin{lstlisting}
				vector<string> names{"daniel", "manuel"};
				
				for (const auto& name: names) {
					cout << names << endl;
				}
				
				\end{lstlisting}
				
				\normalSizeItem { Display \texttt{true}/\texttt{false} with \texttt{bool} and use more \texttt{float} precision }
				\begin{lstlisting}[basicstyle={\tiny\ttfamily}]
				#include <iomanip>
				
				boolalpha(cout); // or: cout.flags(std::ios_base::boolalpha);
				// "noboolalpha(cout)" to disable it
				bool isHidden = false;
				cout << true << ' ' << isHidden << endl; // true false
				
				const long double pi = std::acos(-1.L);
				cout.precision(std::numeric_limits<long double>::digits10 + 1);
				cout << pi << endl; // 3.141592653589793239
				\end{lstlisting}
				
			\end{itemize}
		\end{frame}
		
		\begin{frame}[fragile]{\ttfamily{numeric\_limits} and \ttfamily{initializer\_list}}	
			\subsection{\ttfamily{numeric\_limits} and \ttfamily{initializer\_list}}		
			\begin{itemize}
			
				\normalSizeItem { Numeric limits }
				\begin{lstlisting}[basicstyle={\tiny\ttfamily}]
				// Formerly [<climits>]
				INT_MIN // -2147483648
				LONG_MAX // 9223372036854775807
				
				// Nowadays [<cstdint>] y [<limits>]
				INT32_MIN // -2147483648
				UINT8_MAX // 255
				numeric_limits<uint16_t>::max() // 65535
				numeric_limits<float>::lowest() // -3.40282e+38
				
				\end{lstlisting}
				
				\normalSizeItem { initializer\_list }
				\begin{lstlisting}[basicstyle={\tiny\ttfamily}]
				class vector {
					vector(...) {}
					vector(initializer_list ilist) {}
				...
				
				vector numbers = {1,2,3,4}; // vector
				auto moreNumbers = {1,3,5,7}; // initializer_list
				\end{lstlisting}
				
			\end{itemize}
		\end{frame}
		
		\begin{frame}[fragile]{Random number [C++11]}	
			\subsection{Random number [C++11]}
			\begin{itemize}
			
				\normalSizeItem { Generate a random number in a range }
				\begin{lstlisting}
				#include <vector>
				#include <random>
				
				using namespace std;
				
				int main() {	
					
					vector<int> numbers {1,2,3,4,5,6};
					
					random_device randomDevice;
					mt19937 generator(randomDevice());
					
					uniform_int_distribution<int> randomValue(1, 90);
					cout << randomValue(generator) << endl;
				}
				\end{lstlisting}				
			\end{itemize}
		\end{frame}
		
		\begin{frame}[fragile]{\ttfamily{<algorithm>} functions}	
			\begin{itemize}
			
				\normalSizeItem { Sort, shuffle, fill, copy... }
				\begin{lstlisting}[basicstyle={\tiny\ttfamily}]
				vector<int> numbers {1,2,3,4,5,6};
	
				random_device randomDevice;
				mt19937 generator(randomDevice());
				
				// Possible output: 6 4 1 3 2 5 
				std::shuffle(numbers.begin(), numbers.end(), generator);
				
				numbers = {1,2,3,4,5,6};
				
				vector<int> numbers2 {9,10};
				
				// numbers result: {9,10,3,4,5,6}
				copy(numbers2.begin(), numbers2.end(), numbers.begin());
				
				// Result: 3 4 5 6 9 10
				sort(numbers.begin(), numbers.end());
				
				// What if I want to change the order...?
				\end{lstlisting}				
			\end{itemize}
		\end{frame}
		
	\newSectionWithoutIndex{Functional Programming}	
	
		\begin{frame}[fragile]{[C++11] Lambda expressions}	
			\subsection{[C++11] Lambda expressions}		
			\begin{itemize}
			
				\normalSizeItem { Inline functions as parameters }
				\begin{lstlisting}
				// Result: 10 9 6 5 4 3 
				sort(numbers.begin(), numbers.end(), [](int lhs, int rhs) { 
					return lhs >= rhs; 
				});
				
				\end{lstlisting}	
				
				\normalSizeItem { Syntax }
				\begin{lstlisting}
				[ capture-list ] ( params ) -> ret { body }
				
				capture-list:
				[a,&b] - Captures "a" by copy, "b" by reference
				[this] - Captures the current object value
				[&] - Captures variables by reference
				[=] - Captures variables by copy
				[] - Captures nothing
				\end{lstlisting}	
							
			\end{itemize}
		\end{frame}
		
		\begin{frame}[fragile]{[C++11] Lambda expressions}			
			\begin{itemize}				
				\normalSizeItem { How to pass functions }
				\begin{lstlisting}[basicstyle={\tiny\ttfamily}]
				#include <functional>
				
				double operation(double lhs, double rhs, 
							std::function<double(double,double)> operationFunctor) {
					return operationFunctor(lhs, rhs);
				}
				
				int main() {
				
					double multiply = operation(10, 30.1, [](double lhs, double rhs) {
						return lhs * rhs;
					});
					
					double add = operation(20.1, 70, [](double lhs, double rhs) {
						return lhs + rhs;
					});
					
					double number = 1000;
					double custom = operation(20.1, 70, [&](double lhs, double rhs){
						return (number * lhs) + rhs;
					});
				}
				\end{lstlisting}
				
			\end{itemize}
		\end{frame}
		
		\begin{frame}[fragile]{Filter, Map \& Reduce (evt::Array)\smallCite{Array}}	
		
			\begin{itemize}
				\normalSizeItem { Filter }
				\begin{lstlisting}
				Array<string> names {"Daniel", "John", "Peter"};
				auto filtered = names.filter([](const string& str) {
					return str.size() > 4;
				}); // ["Daniel", "Peter"]
				\end{lstlisting}
				
				\normalSizeItem { Map \& Reduce }
				\begin{lstlisting}
				size_t totalSize = Array<string>({"names", "john"})
				    .map<size_t>([](auto str) {
					    return str.size();
				    }) // [5, 4]
				    .reduce<size_t>([](auto total, auto strSize) {
					    return total + strSize; 
				    }); // 9
				\end{lstlisting}
			\end{itemize}
		\end{frame}

		\newSectionWithoutIndex{Advanced}
	
		\begin{frame}[fragile]{Constant expressions \ttfamily{(constexpr)\smallCite{constexpr}}}	
			\subsection{Constant expressions \ttfamily{(constexpr)}}		
			\begin{itemize}
			
				\normalSizeItem { Functions }
				\begin{lstlisting}
				constexpr uint64_t factorial(int n) {
					return n <= 1 ? 1 : (n * factorial(n - 1));
				}
				
				int main {
					// error: static_assert failed "wrong!"
					static_assert(factorial(2) == 3, "wrong!");
				}
				
				\end{lstlisting}
				
				\normalSizeItem { Variables }
				\begin{lstlisting}
				constexpr uint64_t factorialResult = factorial(2);
				// static_assert(factorialResult == 3, "wrong!");
			
				constexpr uint64_t other = factorialResult * 2;
				static_assert(other == 4, "wrong!");
				\end{lstlisting}
				
			\end{itemize}
		\end{frame}
		
		\begin{frame}[fragile]{Constant expressions \ttfamily{(constexpr)}}		
			\begin{itemize}
			
				\normalSizeItem { Objects }
				\begin{lstlisting}
				class Circle {
					int x_;
					int y_;
					int radius_;
				public:
					constexpr Circle (int x, int y, int radius): 
						x_(x), y_(y), radius_(radius) {}
					constexpr double area() const {
						return radius_ * radius_ * 3.1415926;
					}
					// ...
				};
				
				int main() {
					constexpr auto myCircle = Circle(10,20,30);
					static_assert(myCircle.area() < 3000, "wrong!");
				}
				
				\end{lstlisting}
			\end{itemize}
		\end{frame}
		
		\begin{frame}[fragile]{Constant expressions \ttfamily{(constexpr)}}	
			\begin{itemize}
				\normalSizeItem { Compile-time check }
				\begin{lstlisting}
				#include <type_traits> // is_unsigned, is_object, etc
				
				template <typename Type>
				void doSomething(Type something) {	
					if constexpr (std::is_same<Type, int>()) {
						Type number = something * 10;
						cout << "int!!: " << number << endl;
					}
					else if constexpr (std::is_same<Type, string>()) {
						Type str = something;
						cout << "string!!: " << str.length() << endl;
					}
				}
	
				int main() {
					doSomething(800);
					doSomething("Daniel"s);
				}
				\end{lstlisting}
			\end{itemize}
		\end{frame}
		
		\begin{frame}[fragile]{Optional value}	
			\subsection{Optional value}
			\begin{itemize}
				\normalSizeItem { Sometimes we should not (or we can't) return an specific value }
				\begin{lstlisting}[basicstyle={\tiny\ttfamily}]
				#include <iostream>
				#include <vector>
				#include <experimental/optional>
				
				using namespace std::experimental;
				
				size_t indexOf(int number, std::vector<int> numbers) {
					for (size_t i = 0; i < numbers.size(); i++) {
						if (number == numbers[i]) { 
							return i; 	
						}
					}
					return 0; // It's confusing... what if the valid returned position is 0
				}
				
				std::optional<size_t> safeIndexOf(int number, std::vector<int> numbers) {
					for (size_t i = 0; i < numbers.size(); i++) {
						if (number == numbers[i]) {
							return i;
						}
					}
					return nullopt;
				}
				\end{lstlisting}
			\end{itemize}
		\end{frame}
		
		\begin{frame}[fragile]{Optional value}	
			\begin{itemize}
				\normalSizeItem { The most secure (\& elegant) way is by using an optional value }
				\begin{lstlisting}
				int main() {
	
					vector<int> otherNumbers{};
				
					size_t index = indexOf(100, otherNumbers);
					// Segmentation fault
					// cout << otherNumbers[index] << endl;
					
					vector<int> numbers{1,2,3,4,10};
					if (auto index2 = safeIndexOf(100, numbers)) {
						cout << numbers[*index2] << endl;
						// index2.value_or(0)
					}
				}
				\end{lstlisting}
			\end{itemize}
		\end{frame}
		
		\begin{frame}[fragile]{Optional value}
			\begin{itemize}
				\normalSizeItem { Returned value by an array when you access a position }
				\begin{lstlisting}
				inline optional<Type> at(const size_t index) const {
					if (index >= count_) {
						return nullopt;
					} else {
						return this->values[index];
					}
				}
				\end{lstlisting}
			\end{itemize}
		\end{frame}
		
		\begin{frame}[fragile]{\ttfamily{std::enable\_if<>}}
			\subsection{\ttfamily{std::enable\_if<>}}	
			\begin{itemize}
				\normalSizeItem { Template type constraints }
				\begin{lstlisting}
				#include <type_traits>
				
				template <typename ArithmeticType, 
					typename = typename std::enable_if<
						std::is_arithmetic<ArithmeticType>::value
					>::type>
				ArithmeticType doSomething(ArithmeticType number) {
					return number * 100;
				}
				
				int main() {
				
					doSomething(800); // OK
					
					// candidate template ignored: disabled by 'enable_if'
					doSomething("Daniel"s); // ERROR 
				}
				\end{lstlisting}
			\end{itemize}
		\end{frame}
		
		\begin{frame}[fragile]{Good practices and tips for classes \smallCite{cppbestpractices}}	
			\subsection{Good practices and tips for classes}
			\begin{itemize}
				\normalSizeItem { include/Human.hpp }
				\begin{lstlisting}[basicstyle={\tiny\ttfamily}]
				#pragma once // instead of the classic guards (#ifndef...)
				
				#include <cstdint>
				#include <string_view> // C++17
				// using namespace std; // Don't use in .h, .hpp files

				namespace evt { // HIGHLY recommended
					class Human {
						// Direct variable initialization [c++11]
						// NEVER use underscores at the beginning of a name
						uint8_t age_{};
						std::string_view name_{};
					public:
						// constexpr Human(){} // Might or might not make sense
						constexpr Human(const uint8_t age, const std::string_view& name): age_(age), name_(name) {}
						
						constexpr uint8_t age() const {  
							return this->age_;
						}
						constexpr std::string_view name() const { 
							return this->name_;
						}
					};
				}
				\end{lstlisting}
			\end{itemize}
		\end{frame}
		
		\begin{frame}[fragile]{Good practices and tips for classes}	
			\begin{itemize}
				\normalSizeItem { Main.cpp }
				\begin{lstlisting}[basicstyle={\tiny\ttfamily}]
				#include "include/Human.hpp"
				
				// using namespace evt; (optional)
				
				// Is not advisable to pass complex types by copy...
				// void giveMeTheObject(evt::Human human) { ... }
				
				// ... or by reference (the object could be modified)
				// void giveMeTheObject(evt::Human& human) { ... }
				
				// The best way is by constant reference
				void giveMeTheObject(const evt::Human& human) { ... }
				
				int main() {
					constexpr evt::Human daniel(10, "Daniel");
					static_assert(daniel.name() == "Daniel", "incorrect!");
				}
				\end{lstlisting}
			\end{itemize}
		\end{frame}
		
	\newSectionWithoutIndex{Compilation}	
  
		\begin{frame}[fragile]{Compilation with new C++ versions}	
			\subsection{Compile with C++11 code}
			\begin{itemize}
			
				\normalSizeItem {Add the C++ version after \texttt{-std=}}
				\begin{lstlisting}
				g++ main.cpp -std=c++11
				g++ main.cpp -std=c++14
				g++ main.cpp -std=c++17 // o: 1z
				\end{lstlisting}
				
				\normalSizeItem {\textit{Flags} recomendados}
				\begin{lstlisting}
				-Wall ->	  Enables important warnings
				-Wextra ->	Enables more warnings
				-O1 / -O2 / -Os / -O3 / -fast ->	Optimizations
				~~
				g++ main.cpp -std=c++14 -Wall -Wextra -Os
				\end{lstlisting}
				
			\end{itemize}
		\end{frame}
		
		\begin{frame}[fragile]{\texttt{Makefile} example}	
			\subsection{\texttt{Makefile} example}
			\begin{itemize}
			
				\normalSizeItem {Complete Example}
				\begin{lstlisting}
				## FLAGS ##
				Libraries = -L lib
				Headers = -I include
				Sources = main.cpp $(Headers) $(Libraries)
				CompilerFlags = -std=c++14 -Os -Wall -Wextra
				OutputName = test
				
				## TARGETS ##
				all: 
					@g++ $(CompilerFlags) $(Sources) -o $(OutputName)
				
				clean:
					@rm -i $(OutputName)*
				\end{lstlisting}
				
			\end{itemize}
		\end{frame}
		
	\newSectionWithoutIndex{Reference}	
		\tiny
		\begin{frame}[allowframebreaks]{Reference}
			\bibliographystyle{plainurl}
			\bibliography{References}
			\small
			Suggestions:
			\begin{itemize}
				\urlItem{http://devdocs.io}
				\urlItem{https://tex.stackexchange.com}
			\end{itemize}
		\end{frame}
\end{document}
